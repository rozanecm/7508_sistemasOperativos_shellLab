\documentclass{article}
\usepackage[utf8]{inputenc}
\usepackage[spanish]{babel}
\usepackage{listingsutf8}
\usepackage{xcolor}
\usepackage{pdfpages}
\usepackage{geometry}
\usepackage{hyperref}

\geometry{
    a4paper,
    margin=1.2in
}

\title{Sistemas Operativos (75.08): Lab Shell}
\author{Matias Rozanec (\texttt{\#97404})\\\texttt{rozanecm@gmail.com}}
\date{Abril 2018}

\begin{document}
\maketitle
\newpage

\tableofcontents
\newpage

% *** CONSIGNA ***
% \part{Consigna}
% \includepdf[pages=-]{consigna.pdf}

% *** RESOLUCION ***
% Some settings for coding style
\lstset{
    basicstyle=\linespread{0.9}\ttfamily\footnotesize,
    frame=single,
    frameround=tttt,
    numbers=left,
    numberstyle=\tiny,
    linewidth=14cm,
    literate=
      {á}{{\'a}}1 {é}{{\'e}}1 {í}{{\'i}}1 {ó}{{\'o}}1 {ú}{{\'u}}1
      {Á}{{\'A}}1 {É}{{\'E}}1 {Í}{{\'I}}1 {Ó}{{\'O}}1 {Ú}{{\'U}}1
      {à}{{\`a}}1 {è}{{\`e}}1 {ì}{{\`i}}1 {ò}{{\`o}}1 {ù}{{\`u}}1
      {À}{{\`A}}1 {È}{{\'E}}1 {Ì}{{\`I}}1 {Ò}{{\`O}}1 {Ù}{{\`U}}1
      {ä}{{\"a}}1 {ë}{{\"e}}1 {ï}{{\"i}}1 {ö}{{\"o}}1 {ü}{{\"u}}1
      {Ä}{{\"A}}1 {Ë}{{\"E}}1 {Ï}{{\"I}}1 {Ö}{{\"O}}1 {Ü}{{\"U}}1
      {â}{{\^a}}1 {ê}{{\^e}}1 {î}{{\^i}}1 {ô}{{\^o}}1 {û}{{\^u}}1
      {Â}{{\^A}}1 {Ê}{{\^E}}1 {Î}{{\^I}}1 {Ô}{{\^O}}1 {Û}{{\^U}}1
      {œ}{{\oe}}1 {Œ}{{\OE}}1 {æ}{{\ae}}1 {Æ}{{\AE}}1 {ß}{{\ss}}1
      {ű}{{\H{u}}}1 {Ű}{{\H{U}}}1 {ő}{{\H{o}}}1 {Ő}{{\H{O}}}1
      {ç}{{\c c}}1 {Ç}{{\c C}}1 {ø}{{\o}}1 {å}{{\r a}}1 {Å}{{\r A}}1
      {€}{{\euro}}1 {£}{{\pounds}}1 {«}{{\guillemotleft}}1
      {»}{{\guillemotright}}1 {ñ}{{\~n}}1 {Ñ}{{\~N}}1 {¿}{{?`}}1
}
\part{Resolución}
\section{Parte 1: Invocación de comandos}
Para resolver la primera parte de este Lab, se alteraron las líneas correspondientes al caso de ejecución de comandos en el switch de la función \texttt{void exec\_cmd(struct cmd* cmd)} del archivo \texttt{exec.c} (líneas 6 a 16 del fragmento de código recuadrado).\\

Para lograr invocar programas y permitir pasarles argumentos, lo primero que se hizo fue castear el \texttt{struct cmd* cmd} a \texttt{(struct execcmd*) full\_cmd}; de esta forma logramos tener acceso de forma directa a todos los datos necesarios del comando recibido. Por último, invocamos \texttt{execvp} pasándole como primer argumento \texttt{argv[0]}, o sea el nombre del programa a ejecutar, y como segundo argumento \texttt{argv}, o sea toda la lista de argumentos.\\

Para poder expandir variables de entorno, lo que primero que se hizo fue trabajar sobre la detección de las mismas. Esto se implementó leyendo el primer caracter de cada uno de los argumentos. En caso de que el caracter coincida con \texttt{\$}, se trata de una variable de entorno, por lo que se procede a su expansión mediante \texttt{getenv()}. A dicha función hay que pasarle el argumento correspondiente pero sin el caracter \texttt{\$}.

\lstinputlisting[firstline=58, lastline=107,caption={exec.c}]{code/part1/exec.c}
\lstinputlisting[firstline=85, lastline=109,caption={parsing.c}]{code/part1/parsing.c}

\newpage
\section{Parte 2: Invocación avanzada}
\subsection{Comandos \textit{built-in}}
Pregunta: ¿entre cd y pwd, alguno de los dos se podría implementar sin necesidad de ser built-in? ¿por qué? ¿cuál es el motivo, entonces, de hacerlo como built-in? (para esta última pregunta pensar en los built-in como true y false)\\

Respuesta: Ambos comandos podrían ser implementados sin necesidad de ser built-in. La razón por la que no se opta por esa opción es que son comandos muy simples que no requieren el tratamiento más comlejo que otros comandos sí necesitan, y por ende se espera una ejecución rápida sin procesamiento innecesario. 
\lstinputlisting[caption={builtin.c}]{code/part2/builtin.c}

\subsection{Variables de entorno adicionales}
Pregunta: ¿por qué es necesario hacerlo luego de la llamada a \texttt{fork(2)}?\\

Es necesario hacerlo luego de la llamada a \texttt{fork(2)} porque de esa forma la incorporación de las nuevas variables sucede en el proceso hijo, que es lo que nos interesa.\\

Responder (opcional):
\begin{itemize}

    \item \textit{¿el comportamiento es el mismo que en el primer caso? Explicar qué sucede y por qué.}\\
        El comportamiento no será el mismo, ya que en el primer caso la variable se agrega al conjunto de variables de entorno, y luego el comando a ejecutar se busca en todo el entorno. En el segundo caso, en cambio, se busca el comando únicamente en el entorno descripto por el array que se le pasa como tercer argumento.
    \item \textit{Describir brevemente una posible implementación para que el comportamiento sea el mismo.}\\
        Una posible implementación sería armar un array con todo el env default, agregándole nuestras nuevas variables, y pasar eso como tercer argumento.
\end{itemize}
\lstinputlisting[firstline=26, lastline=43,caption={exec.c: funcion set\_environ\_vars}]{code/part2/exec.c}
\subsection{Procesos en segundo plano}
\textit{Detallar cúal es el mecanismo utilizado.}\\
En caso de detectar un proceso que deba ser corrido en \textit{background}, no se espera a que el hijo termine. Nótese que se diversifica el print de estado de acuerdo al tipo de proceso que se está corriendo. El free del comando se hace en cualquier caso, de lo contrario habrá leaks. de memoria. 
\lstinputlisting[firstline=44, lastline=67,caption={runcmd.c: últimas líneas}]{code/part2/runcmd.c}

\subsection{Pregunta de parte 1}
Respondo a continuación la pregunta de la parte 1 que para la primer entrega pasé por alto.\\

\textit{Pregunta: ¿cuáles son las diferencias entre la syscall execve(2) y la familia de wrappers proporcionados por la librería estándar de C (libc) exec(3)?}\\
La syscall recibe como argumentos el nombre de programa a ejecutar, un array de argumentos que se le pasarán al nuevo programa, y array especificando el ambiente del programa. La familia de wrappers llama internamente a la syscall, pero hacen que sea más simple el manejo de la misma, contemplando los diversos escenarios posibles.


\newpage
\section{Parte 3: Redirecciones}
\subsection{Flujo estándar}
El esqueleto dado se encarga en su etapa de parsing de almacenar los nombres de archivo a los cuales habría que redireccionar el flujo de salida estándard y error, o de los cuales habría que tomar la entrada estándard. Gracias a eso, lo único que restaba hacer es detectar si se trataba de un comando de tipo REDIR, en cuyo caso se crea el archivo correspondiente (en caso de redireccionar salidas) o se abre el archivo solicitado para lectura. En cualquiera de los casos, se le cambia el file descriptor a estos archivos con los de las entradas y salidas estandar, mediante \texttt{dup2()}.\\

En el caso de la redirección \texttt{2\textgreater\&1}, se chequea si el archivo al que se quiere redireccionar la salida de error tiene `nombre' == `\texttt{\&1}', en cuyo caso se redirecciona la salida de error a la salida std. \\

\textit{Investigar el significado de este tipo de redirección y explicar qué sucede con la salida de cat out.txt. Comparar con los resultados obtenidos anteriormente.}\\
\texttt{2\textgreater\&1} redirecciona la salida de error a la salida std. \texttt{\textgreater out.txt} redirecciona la salida std al archivo \textit{out.txt}. Por lo tanto, todo lo que saldría por la salida std y por la salida de error, se guardará en el archivo \textit{out.txt}.
\lstinputlisting[firstline=98, lastline=138,caption={exec.c: líneas correspondientes a manejo de redirecciones.}]{code/part3/redir_exec.c}

\textit{\textbf{Challenge:} investigar, describir y agregar la funcionalidad del operador de redirección $>>$ y \&\textgreater}\\

El operador \texttt{$>>$} escribe en el archivo cuyo nombre le sucede a dicho operador, pero en lugar de crear un archivo nuevo y escribir desde el principio, en caso de que dicho archivo ya exista, lo abre en modo \textit{append}. \\

El operador \texttt{\&\textgreater} redirige las salidas std y de error al archivo cuyo nombre le sucede al comando.
En el código a continuación, se puede ver que en el caso de haber redireccionamiento de output, se chequea si se usó el operador \texttt{$>>$}, en cuyo caso se abre el archivo en modo append.
\lstinputlisting[firstline=97, lastline=145,caption={exec.c: líneas correspondientes a manejo de redirecciones.}]{code/part3/comb_exec.c}
Para la implementación del comando \texttt{\&\textgreater} hubo que corregir primero el manejo de los comandos BACK:
\lstinputlisting[firstline=89, lastline=95,caption={exec.c: líneas corrgidas del manejo de comandos BACK.}]{code/part3/comb_exec.c}
Luego, en el archivo \textit{parsing.c} se corrigió el código para que no se entienda como un comando BACK en caso de encontrar la secuencia \texttt{\&\textgreater}. 
\lstinputlisting[firstline=171, lastline=189,caption={parsing.c: se agrega caso de excepción, evitando malinterpretar comandos como BACK.}]{code/part3/parsing.c}
Por último, en \texttt{parse\_redir\_flow()} de \textit{parsing.c} se consideró el caso de que se quiera redireccionar las salidas std y de error a un archivo (adentro del \texttt{case: 1}).
\lstinputlisting[firstline=20, lastline=60,caption={parsing.c: de detectrse la secuencia \&\textgreater, se guarda el nombre de archivo tanto para la salida std como de error.}]{code/part3/parsing.c}

\subsection{Tuberías simples (pipes)}
En este caso se completó el código correspondiente al caso de PIPE en la función \textit{exec\_cmd} del archivo \textit{exec.c}. En primer lugar se llama a \texttt{pipe()} para obtener dos file descriptors. A continuación, se hace un fork: el proceso hijo redirecciona el stdout al segundo file descriptor dado por \texttt{pipe()}, y luego llama a \textit{exec\_cmd()} con el comando de la izquierda del pipe. El roceso padre redirecciona el stdin al primer file descriptor dado por \texttt{pipe()} y llama a \textit{exec\_cmd} con el comando de la derecha del pipe.
\lstinputlisting[firstline=147, lastline=188,caption={exec.c: PIPE case}]{code/part3/pipe_exec.c}

\newpage
\section{Challenges promocionales}
\subsection{Pseudo-variables}
Para implementar el correcto funcionamiento de la pseudo variable \texttt{?}, se hizo llegar el status number hasta la expansión de variables, donde de tener que expandir la variable \texttt{?}, se la expande poniendo en su lugar el \textit{exit status of the most recently executed foreground pipeline}. Para que el status le llegue a la función correspondiente, fue necesario pasarlo a la función \texttt{parse\_line()}, que se lo pasa a \texttt{parse\_cmd()}, que se lo pasa a \texttt{parse\_back()} y \texttt{parse\_exec()}, y esta última se lo pasa finalmente a \texttt{expand\_environ\_var()}.

\lstinputlisting[firstline=98, lastline=123,caption={parsing.c: modificación de exapnsión de variables.}]{code/challenges/parsing.c}

\textit{Investigar al menos otras dos variables mágicas estándar, y describir su propósito.}\\

Una lista completa de las variables mágicas estándar y sus propósitos se pueden encontrar en el \href{https://www.gnu.org/software/bash/manual/html\_node/Variable-Index.html}{Bash reference manual.} Se presentan a continuación dos de ellas, según solicitado.\\

\begin{description}
    \item[\$] Expande al process ID de la shell. En una () subshell, expande al process ID de la shell invocante, no de la subshell.
    \item[0] Expande al nombre de la shell o shell script. Esto es un set en el momento de la inicialización de la shell. Si Bash es invocado con un archivo de comandos, \$ es seteado al onmbre de ese archivo. Si Bash es ejecutado con la opción \texttt{-c}, \$ es seteado al primer argumento después del string a ejecutar, si hay alguno presente. De lo contrario, es seteado al filename usado para invocar Bash, dado por el argumento cero.
\end{description}

\subsection{Tuberías múltiples}
Para implementar las tuberías múltiples, se chequea al momento de parsear un comando si el mismo contiene un caracater pipe, en cuyo caso se retorna lo que retorne \texttt{parse\_line(comando derecho de pipe)}. Nótese que esta implementación permite agregar la cantidad de pipes que uno desee sin inconvenientes.

\lstinputlisting[firstline=179, lastline=200,caption={parsing.c: ver lineas 8 y 9.}]{code/challenges/pipe_parsing.c}

\subsection{Segundo plano avanzado}
La clave para resolver este challenge fue el manejo de señales. Cada vez que un proceso termina, sea naturalmente o por interrupción, éste lanza la señal \texttt{SIGCHLD}. Mediante la función  \texttt{sigaction()} se llama a una función \texttt{handler()} antes de escribir nuevamente el \textit{prompt} cada vez que se recibe una señal de ese tipo, para poder dar el mensaje de proceso terminado. Como este comportamiento es desado únicamente cuando se termina de ejecutar un proceso \textit{background}, se discriminó cada caso chequeando el \textit{pid} del proceso terminante. Esto se logra guardando el \textit{pid} del background process en ejecución, y comparándolo con el pid guardado en la señal. 

\lstinputlisting[caption={sh.c: ver lineas 9 a 14 y 22 a 25.}]{code/challenges/background/sh.c}
\lstinputlisting[firstline=47, lastline=62,caption={runcmd.c: en caso de tratarse de un background process, se asigna el pid a la variable global \texttt{background\_process\_id.}}]{code/challenges/background/runcmd.c}
\end{document}
